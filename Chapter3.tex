\chapter{Dynamic Author-Topic Model}
\label{chapterlabel3}

Chapter~\ref{chapterlabel2} describes the topic models and applications related to our Dynamic Author-Topic model. In Chapter~\ref{chapterlabel3} we will give detailed description of Dynamic Author-Topic model.

\section{Motivation}

The question of ``What is this news article about?" is what we want to answer from our proposed model. At the moment the news on BBC websites are tagged which is used to indicate the cluster of the news, for example, Politics\footnote{http://www.bbc.co.uk/news/politics}, Education\footnote{http://www.bbc.co.uk/news/education}, UK\footnote{http://www.bbc.co.uk/news/uk} and Sport\footnote{http://www.bbc.co.uk/sport} etc. However, the granularity or topic here is so coarse that the hidden relations among news and semantic content of a news cannot be specified, and also manual tagging news is so expensive and not practical for a large dataset of news. 

In order to structure the unstructured news better we can use topic models to discover the hidden thematic structure of the news and cluster the news collection in a better organization, so that when new data (query or news document) arrives it can be properly fitted into the estimated topic structure. LDA is one possible solution to this problem that models the relation of word-topic, topic-document correctly, which will be beneficial for news classification and clustering. 

However, most of the topic models, such as LDA, and Author-Topic model are under the assumption that the documents are exchangeable in the corpus, which means that its probability is invariant to permutation. 

News, is a rolling stream of stories - by definition they live in the present, and evolve over time. So one key challenge we want to solve is, instead discovering static latent topics, we consider news as a continuously streaming of a sequence of documents, its topic distribution change with time with previously salient topics fading-off. For example, on the WAT application of BBC News Lab\footnote{http://wat.bbcnewslabs.co.uk/}, which is a tool to compare coverage on different topics across different media sources using data from the BBC News Labs Juicer, we can obviously see the evolution of ``hot topics" on the media. When searching for UK-related news on BBC, obviously we can see the topic distribution of UK news differ from month to month, \textit{murder} appears in May's top topics because of the death of care worker Saima Khan on 23 May in Luton with her sister on \textit{murder} charge. And in June media was focusing on the referendum which is a vote held on Thursday 23 June, to decide whether the UK should leave or remain in the European Union. And British turns to UEFA Euro 2016 in July with D-day approaching and start to welcome the Rio Olympics from middle of July.
\begin{figure}[h]
\centering
\includegraphics[width=\textwidth]{figures/BBC_wat.png}
\caption{Example of the topic evolution on BBC news: who's talking about what?}
\label{fig:bbc_wat}
\end{figure}

The other challenge is to apply Author-Topic (AT) model for the news collection which simultaneously model the content of the document and also the interest of the authors. The advantage of AT model is that besides representing each news with a mixture of topics, the \textit{author} is also modeled by determining the mixture weights for different topics for a single news, each \textit{author} is associated with a multinomial distribution over topics. This generative model perfectly fits in the corpus of BBC news, since for BBC news there is a virtual \textit{author} for each of the news, which is its category, as shown in~\ref{tab:news_category}. For news in different categories their prose style will differ in the aspects of vocabulary use, sentence structure, as well as the way in which stories present the information in terms of relative importance, tone, and intended audience. 

Thus BBC news can be assumed to be generated in the following way, as shown in Figure~\ref{fig:author_diagram}.

\begin{enumerate}
   \item For each topic $k \in [1,K]$
   \begin{enumerate}
     \item Draw a multinomial $\vec{\phi_k}$ from a Dirichlet prior $\vec{\beta}$
    \end{enumerate}
   \item For each author $a \in [1,A]$
   \begin{enumerate}
     \item Draw a multinomial $\vec{\theta_a}$ from a Dirichlet prior $\vec{\alpha}$
    \end{enumerate}
    \item For each news $m \in [1,M]$
   \begin{enumerate}
     \item For each word $n \in [1,N_m]$ in document $m$
     \begin{enumerate}
            \item Draw an author $x_{m,n}$ uniformly from the group of authors $a_M$
            \item Draw a topic assignment $z_{m,n}$ from per-author multinomila distribution over topic $\vec{\theta}_{x_{m, n}}$ %$\vec{\theta_{x_{m,n}}}$
            \item Draw a word $w_{m,n}$ from multinomial $\vec{\phi}_{z_{m, n}}$
            %$\vec{\phi_{z_{m,n}}}$
    \end{enumerate}
    \end{enumerate}
        
\end{enumerate}

In the above process the posterior distribution of topics are dependent on the information form the authors as well as the text of the news. The parameterization of the AT model can be seen in Section~\ref{Author-Topic model}.

\begin{figure}[h]
\centering
\includegraphics[width=\textwidth]{figures/author_diagram.png}
\caption{Illustration of Author-Topic model on BBC news documents}
\label{fig:author_diagram}
\end{figure}


Therefore in order to encompass the extra-feature of the news, namely its category, in our model, the AT model is used as the basis of our dynamic model to overcome the above two challenges. Here we define the concepts of \textbf{\textit{topic}} and \textbf{\textit{author}} as follows,
\begin{itemize}
  \item A \textbf{\textit{topic}} is a subject discussed in one or more news. Examples of topics include events such as ``UK referendum" entities such as ``David Cameron" and long-standing subjects such as ``UK Politics". Each topic is assumed to be represented by acmultinomial distribution of words.
  \item An \textbf{\textit{author}} is a category of BBC news which groups topics belonging to a common subject
  area together. Examples of authors include BBC news categories of ``UK", ``World", ``Eduction", ``Health", etc, as listed in Table~\ref{tab:news_category}
  \end{itemize}
The model we propose here is Dynamic Author-Topic (DAT) model which draws upon the strength of Author-Topic model and Dynamic Topic model. Considering the temporal nature of news we assume the topic distributions of news will evolve over time frames (a day, a week or a month), and the inferred topic distribution in the past news documents will become the evidence of those for the current new. In each time frame, we assume the news are generated based on Author-Topic model. Information about which topics are associated to which author and the representation of the content of each news document in terms of topics are derived and used for the next time frame. In Section~\ref{taskdescription} we will discuss what kind of problem we want to solve using our proposed model. The details of the proposed model is further represented in Section~\ref{dynamicauthortopicmodel}. Lastly we present how to infer the update rules for the hyperparameters as well as the multinomila parameters of our model using Collapsed Gibbs sampler in Section~\ref{inferenceofthemodel}.

\section{Task Description}
\label{taskdescription}
The task to be addressed in this thesis can fall in the following three aspects:

\begin{itemize}
  \item \textbf{Tagging}: Based on the content and time of the news we are able to automatically tag the news with an author (category)
  \item \textbf{Summarization}: Based on the author-topic distribution we are able to what topics are most discussed in one type of news
  \item \textbf{Dynamics}: We are able to monitor the changes of event of interest for the news over time
\end{itemize}

The input data of our model will be the BBC news text in a range of time, as shown in Figure~\ref{fig:input}.

\begin{figure}[h]
\centering
\includegraphics[width=\textwidth]{figures/input.png}
\caption{Input data: BBC news over time}
\label{fig:input}
\end{figure}
The output result will satisfy the above three aspects, as shown in Figure~\ref{fig:output}.
\begin{figure}[h]
\centering
\includegraphics[width=\textwidth]{figures/output.png}
\caption{Output result: temporal evolution of topics and authors’ proportion over the topics }
\label{fig:output}
\end{figure}



Mathematically, the author-topic and topic-words distribution are changing dynamically with news stream in, our dynamic model is essentially a function $f$ that satisfies:
\begin{equation}
\label{eq:task}
\mathbf{m}_{\le t}=\{\ldots, \mathbf{m}_{t-2}', \mathbf{m}_{t-1}', \mathbf{m}_t'\}
\stackrel{f}{\longrightarrow} \boldsymbol W'_{\boldsymbol a,\boldsymbol k}_{\le t}=\{\{\mathbf{w}_1', \mathbf{w}_2', \ldots, \mathbf{w}_K'\},\{\mathbf{w}_1', \mathbf{w}_2', \ldots, \mathbf{w}_A'\}\},
\end{equation}
where $\mathbf{m}_{\le t}$ represents news stream with $\mathbf{m}_t'$ being the most recent \textit{set} news, arriving at time $t$, with our dynamic model it results in  $\boldsymbol W'_{\boldsymbol a,\boldsymbol k}$ is the resulting set of 3-tuple, $\{w,a,k\}$, which represents word $w$ and the author $a$ and topic $k$ assigned to it, with $\mathbf{w}_k'$ being the words set associated with topic $k$,  with $\mathbf{w}_a'$ being the words set associated with author $a$. $\mathbf{m}'_t$ comprises a the stream of news at time $t$, with each news $m$ being represented by a sequence of words appearing in $m$, coming from the vocabulary $\boldsymbol w$. % and $t$ is the creation time of the document $d$.
%$\mathbf{d}'_t$ comprises a set of short text documents, with each document being represented by a tuple $\langle \mathbf{w}_d, t \rangle$ where $\mathbf{w}_d$ is a sequence of words appearing in document $d$, coming from a vocabulary $\mathbf{V}=\{ v_1, v_2, \ldots, v_V\}$, and $t$ is the creation time of the document $d$. 

Based on $\boldsymbol W'_{\boldsymbol a,\boldsymbol k}$ we are able to calculate the author-topic and topic-word distribution to see the topic evolution over time.
Table~\ref{tab:notation-des} summarises the main notations that we have used in our dynamic author-topic model.
\begin{table}[h]
\center
\vspace{-1pt}
\caption{Notation used in the Dynamic Author-Topic model}
\label{tab:notation-des}
\small
\begin{tabular}{ll}
%\toprule
Symbol & Description\\
\hline

$\bs{K}$ & Number of latent topics\\
$\bs{M}$ & Number of news\\
$\bs{A}$ & Number of unique authors\\
$\bs{A_m}$ & Number of authors in document $m$\\
$\bs{V}$ & Number of unique word tokens in the whole news corpus\\
$T$ & Number of time frames  \\
$N_{\text{m}}$ & Number of word tokens in news m\\
$k$ & Topic index,  $k \in [1,\bs{K]$ \\
$m$ & News index,  $m \in [1,M]$ \\
$a$ & Author index,  $a \in [1,\bs{A}]$ \\
$t$ & Time frame index, $t \in [1,T]$\\
$n$ & Word index in document $m$,  $n \in [1,N_{\text{m}}]$ \\
$v$ & Word index in the whole news document corpus,  $v \in [1,V]$ \\
$\boldsymbol a$ & Set of all authors \\
$\boldsymbol k$ & Set of all topics \\
$\boldsymbol m$ & Set of all news \\
$\boldsymbol m'_t$ & Set of news at time t\\
$\boldsymbol w$ & Words set for all news \\
$\boldsymbol w'_{a}$ & Words set comprising words associated with author $a$ \\
$\boldsymbol w'_{k}$ & Words set comprising words associated with topic $k$ \\
$\boldsymbol W'_{\boldsymbol a,\boldsymbol k}$ & The set of 3-tuple, $\{w,a,k\}$, which represents word $w$ and the author $a$ \\
& and topic $k$ assigned to it \\
$\boldsymbol w_m$ & Words set for news $m$ \\
$\alpha_{t}$ & Parameter of topic Dirichlet prior to $\theta$ at time $t$ \\
$\beta{t}$ & Parameter of word Dirichlet prior to $\phi$ at time $t$ \\
$a_m$ & Authors in news m,  $a_m \in [1,\bs{A]$ \\
$\theta_{a,t}$ & Dynamic multinomial distribution of topics specified to author $a$ at time $t$ \\
$\phi_{k,t}$ & Dynamic multinomial distribution of words specified to topic $k$ at time $t$ \\
$x_{m,n}$ & Author associated to $w_{m,n}$ \\
$z_{m,n}$ & Topic associated to $w_{m,n}$ \\
$w_{m,n}$ & $n_{th}$ word in doc $n$ \\
%\bottomrule

\hline
\end{tabular}
\end{table}

\section{Dynamic Author-Topic Model}\label{dynamicauthortopicmodel}
To meet the requirements discussed in~\ref{taskdescription},  the event interests of the news can be sequentially inferred using the newly published news. For the news its hot topics and trends change over time but writing styles of the news of the same category will keep stable. So that we propose the Dynamic Author-Topic (DAT) model here which is adaptable to changes and also is an incremental model in which updates can be achieved. Our model is able to reflect the news' reality of rapid change and has the capability to deal with accumulated huge amount of data in the real online system.
\subsection{Preliminaries}
The target of the DAT model is to infer the dynamically changing topic-word distribution and author-topic distribution at any given time $t$ with continuously incoming news stream. Therefore, we need to infer the temporal word, $v$'s probability for a topic $k$, $P(v|t, k)$, and the temporal topic probability over an author $a$, which in our case is the news' category, $P(k|t, a)$. Then we can obtain,
\begin{equation}
p(v|t,a) = \sum_{k=1}^K{P(v|t, k)P(k|t, a)},
\end{equation}
which represents the association between the word $v$ and the author $a$, which can be used to automatically classify the news into BBC news category.
\def \thetadef {$\boldsymbol{\Theta}_{a,t}=\{\theta_{a,t, k}\}_{k=1}^K$}
\def \phidef {$\boldsymbol{\Phi}_t=\{\phi_{t, k}\}_{k=1}^K$}

Based on the probabilistic model, \thetadef is the topic distribution associated with author $a$, at time $t$ with  $\sum_{a=1}^A\sum_{k=1}^K\theta_{a, t, k}=1$. We also let \phidef be the words distribution over topics at time $t$. $\phi_{t, k}=\{\phi_{t, k, v}\}_{v=1}^V$ is the multinomial distribution of the words for topic $k$ at time $t$, and also the probability of a word $v$ belonging to $k$ at $t$, $\phi_{t, k, v}=P(v|t, k)>0$, and $\sum_{v=1}^V \phi_{t, k, v}=1$. There is an underlying assumption of the fully bayesian non-dynamic topic models that the topic distribution over an author is independent of the past trends of the news, with a Dirichlet prior with a static set of parameters $\kappa=\{\kappa_k\}_{k=1}^K$, with $\kappa_{k}>0$,
\begin{equation}
\label{eq:assume1}
P(\boldsymbol{\Theta}_{a,t} | \kappa)  \propto \prod_{k=1}^K \theta_{a, t, k}^{\kappa_k -1}.
\end{equation}

Since the news are actually a continuous stream data, therefore the assumptions made in~\eqref{eq:assume1} are not always realistic. So two assumptions here are made for our temporal dependent model. Firstly the means of the author-topic distribution at the moment are the same as those at a previous time point unless otherwise confirmed by the newly coming-in data. Secondly, we assume that in a short time-period the trends and topics of the news remain static, which means that the author-topic distribution at time $t$ will remain the same as that at time $t-1$ if no news is observed, and it will be updated when new set of documents come in at time $t+1$. Hence, the Dirichlet parameter in \eqref{eq:assume1} can be factorized into the mean of the distribution at the previous time-step, $\theta_{a,t-1,k}$ and a set of prevision value, $a = \{a_{t,k}\}_{k=1}^K$, representing the topic persistency on topic $a$. Hence, $\kappa=\alpha_t \boldsymbol{\Theta}_{a,t-1}$, which allows the mean of the current distribution $\boldsymbol{\Theta}_{a,t}$ to depend on the mean of the previous distribution $\boldsymbol{\Theta}_{a,t-1}$.

\begin{equation}
\label{eq:shortTheta}
P(\boldsymbol{\Theta}_{a,t} | \boldsymbol{\Theta}_{a,t-1}, \alpha_t) \propto \prod_{k=1}^K \theta_{a,t, k}^{(\alpha_{t, k} \theta_{a,t-1, k}) -1},
\end{equation}

And similarly, the word distribution over per topic, $\phi_{t, k}$ also has a Dirichlet prior with a static set of parameters $\gamma=\{\gamma_v\}_{v=1}^V$, with $\gamma_v>0$, 
\begin{equation}
\label{eq:assume2}
P(\boldsymbol{\phi}_{t, k} | \gamma) \propto \prod_{v=1}^V \phi_{t, k, v}^{\gamma_v -1},
\end{equation}
the Dirichlet prior parameter $\gamma$ in above can be also factorized into the mean and precision, $\gamma=\beta_{t, k} \phi_{t-1, k}$,  $\beta_t=\{\beta_{t, k}\}_{k=1}^K$ are the set of precision values at time $t$ for the topics. Here $\beta_{t, k, v}$ represents the persistency of word $v$ in topic $k$ at time $t$, used to measure how consistently word $v$ belongs to topic $k$ at time $t$ compared to that at the previous time $t-1$. Therefore, we can obtain another dependency distribution formula:

\begin{equation}
\label{eq:shortPhi}
P(\phi_{t, k} | \phi_{t-1, k}, \beta_{t, k}) \propto \prod_{v=1}^V \phi_{t, k, v}^{(\beta_{t, k, v} \phi_{t-1, k, v}) -1}.
\end{equation}

The distributions in~\eqref{eq:shortTheta} and~\eqref{eq:shortPhi} are conjugate priors of the Multinomial distribution, thus the inference can be performed by Gibbs sampling~\cite{liu1994collapsed} with details showing in Section~\ref{inferenceofthemodel}.
%
\subsection{Description of the Model}


In Section \ref{taskdescription} we have presented how we model our Dynamic Author-Topic model by considering the temporal factor. In this section we will further describe our model in detail. 

\begin{figure}[h]
\centering
\includegraphics[width=1.1\textwidth]{figures/ATOT_Graphic.png}
\caption{Graphical representation of our Dynamic Author-Topic model (DAT) model. Note that short term dependency DAT model excludes the two blue curved lines, whereas the long term dependency DAT model includes the two lines.}
\label{fig:atot}
\end{figure}


Figure~\ref{fig:atot} illustrates the graphical representation of the Dynamic Author-Topic model, on conditional that we have computed the topic distribution over authors at time $t-1$, $\boldsymbol{\Theta}_{a,t-1}$, and the word distribution over topics at time $t-1$, $\boldsymbol{\Phi}_{t-1}$, as well as the hyperparameters $\alpha_{t-1}$ and  $\beta_{t-1}$. The parameterization of our proposed model for news stream  at time $t$, is as follows,

\begin{eqnarray*} \label{eq:dat}
\boldsymbol{\Theta}_{a,t} | \boldsymbol{\alpha_{a,t}}
\boldsymbol{\Theta}_{a,t-1}
& \sim & \text{Dirichlet}({\boldsymbol{\alpha_{a,t}}
\boldsymbol{\Theta}_{a,t-1}})\\
\boldsymbol{\Phi_{k,t}} | \boldsymbol{\beta_{k,t}}\boldsymbol{\Phi_{k,t-1}} & \sim & \text{Dirichlet}(\boldsymbol{\beta_{k,t}}\boldsymbol{\Phi_{k,t-1}})\\
z_{m,n} | \boldsymbol{\Theta_{x_{m,n},t}} & \sim & \text{Multinomial}(\boldsymbol{\Theta_{x_{m,n},t}})\\
w_{m,n} | \boldsymbol{\Phi_{z_{m,n},t}} & \sim & \text{Multinomial}(\boldsymbol{\Phi_{z_{m,n},t}})\\
x_{m,n} | {A_{m}} & \sim & \text{Multinomial}(1/A_m)\\

\end{eqnarray*}

As in Author-Topic Model each topic $z$ is sampled from an author-specific multinomial distribution and each word $w$ is sampled from a topic-specific multinomila distribution.

The generative process of our model on the news stream at time $t$ is as follows,
\begin{enumerate}
   \item For each topic $k \in [1,K]$
   \begin{enumerate}
     \item Draw a multinomial $\vec{\phi}_{t,k}$ from a Dirichlet prior $\vec{\beta}_{t,k}\vec{\phi}_{t-1,k}$
    \end{enumerate}
   \item For each author $a \in [1,A]$
   \begin{enumerate}
     \item Draw a multinomial $\vec{\theta}_{a,t}$ from a Dirichlet prior $\vec{\alpha}_{a,t}\vec{\theta}_{t-1,a}$
    \end{enumerate}
    \item For each news $m \in [1,M]$
   \begin{enumerate}
     \item For each word $n \in [1,N_m]$ in document $m$
     \begin{enumerate}
            \item Draw an author $x_{m,n}$ uniformly from the group of authors $a_M$
            \item Draw a topic assignment $z_{m,n}$ from per-author multinomila distribution over topic $\vec{\theta}_{x_{m, n},t}$ %$\vec{\theta_{x_{m,n},t}}$
            \item Draw a word $w_{m,n}$ from multinomial $\vec{\phi}_{z_{m, n},t}$
            %$\vec{\phi_{z_{m,n},t}}$
    \end{enumerate}
    \end{enumerate}
        
\end{enumerate}


Since the inference of the distribution of our proposed model is intractable, therefor Gibbs sampler as discussed in \cite{wang2006topics} is used for parameter estimation, we adopt a Dirichlet prior for the multinomial distribution which is its conjugate, thus the uncertainty associated with $\boldsymbol{\Phi_t}$ and $\boldsymbol{\Theta_t}$ can be easily integrated out. In the Gibbs sampling procedure our target is to calculate the conditional distribution at time $t$, which is
$P({z}_{m, n, t},{x}_{m, n, t}| \mathbf{w}_t,\mathbf{z}_{\neg(m, n, t)}, \mathbf{x}_{\neg(m, n, t)},\mathbf{a_t},\alpha_t, \beta_t,\boldsymbol{\Phi}_{t-1}, \boldsymbol{\Theta}_{t-1})$.
The process of model inference is discussed in Section~\ref{inferenceofthemodel}. Briefly, using the chain rule we can finally obtain the following joint probability in \eqref{joint} , 
\begin{align}\label{joint}
\multicolumn{2} =   &  \math{P}({z}_{m, n, t},{x}_{m, n, t}| \mathbf{w}_t,\mathbf{z}_{\neg(m, n, t)}, \mathbf{x}_{\neg(m, n, t)},\mathbf{a_t},\alpha_t, \beta_t,\boldsymbol{\Phi}_{t-1}, \boldsymbol{\Theta}_{t-1}) \nonumber
\displaybreak[3]\\  \nonumber
\propto & \quad  \left(\frac{n_{t,v,k}+\beta_{t,k,v}\phi_{t-1}-1}{\sum_{v=1}^V \left(n_{t,v,k}+\beta_{t,v,k}\phi_{t-1} \right)-1 } \right) \times   \left(\frac{n_{t,a,k}+\alpha_{t,k} \Theta_{t-1}-1}{\sum_{k=1}^K \left(n_{t,a,k}+\alpha_{t,k}\theta_{t-1} \right)-1 } \right).
\displaybreak[3]\\
\end{align}
 
The joint probability in \eqref{joint} can be then turned  into the separated update rules for associated topic and author of a work token. The update rule for topic is shown in \eqref{topic_update_rule} and the one for the author is \eqref{author_update_rule},


\begin{align}\label{topic_update_rule}
\multicolumn{2} =   &  \math{P}({z}_{m, n, t}| \mathbf{w}_t,\mathbf{z}_{\neg(m, n, t)},\mathbf{x}_t, \mathbf{a_t},\alpha_t, \beta_t,\boldsymbol{\Phi}_{t-1}, \boldsymbol{\Theta}_{t-1}) \nonumber
\displaybreak[3]\\ \nonumber
\propto & \quad  \left(\frac{n_{t,v,k}+\beta_{t,k,v}\phi_{t-1}-1}{\sum_{v=1}^V \left(n_{t,v,k}+\beta_{t,v,k}\phi_{t-1} \right)-1 } \right) \times   \left(\frac{n_{t,a,k}+\alpha_{t,k} \Theta_{t-1}-1}{\sum_{k=1}^K \left(n_{t,a,k}+\alpha_{t,k}\theta_{t-1} \right)-1 } \right),
\displaybreak[3]\\
\end{align}
  
\begin{equation}\label{author_update_rule}
 \math{P}({x}_{m, n, t}| \mathbf{z}_t,\mathbf{x}_{\neg(m, n, t)}, \mathbf{a_t},\alpha_t,  \boldsymbol{\Theta}_{t-1}) \quad
\propto  \quad   \left(\frac{n_{t,a,k}+\alpha_{t,k} \Theta_{t-1}-1}{\sum_{k=1}^K \left(n_{t,a,k}+\alpha_{t,k}\theta_{t-1} \right)-1 } \right).
\displaybreak[3]\\
\end{equation}

 \eqref{topic_update_rule} and~\eqref{author_update_rule} are used to update $x_{m,n}$ and $z_{m,n}$ for each iteration, by maximizing the joint distribution~\eqref{joint} the precision parameters can be estimated, the fixed point iteration is applied to get the optimal $\alpha$ and $\beta$ at time $t$ and the following update rules are obtained with details in Section~\ref{inferenceofthemodel}. 
 \begin{equation}\label{dat_alpha1}
 \quad \alpha^*_{t,k} \gets \frac{\sum_{a=1}^A \alpha_{t, k}(\psi(n_{a,t,k}+\alpha_{t, k} \theta_{t-1, k})-\psi(\alpha_{t,k}\theta_{t-1, k}))}{\sum_{a=1}^A (\psi(\sum_{k=1}^K(\alpha_{t, k} \theta_{t-1, k}+n_{t,a,k}))-\psi(\sum_{k=1}^K(\alpha_{t, k} \theta_{t-1, k})))}, \displaybreak[3]\\
\end{equation}

\begin{equation}\label{eq:beta1}
 \beta^*_{t,k,v} \gets \frac{\sum_{k=1}^K \beta_{t,v,k}\phi_{t-1}(\psi(\beta_{t,k,v}\phi_{t-1}+n_{t,v,k})-\psi(\beta_{t,v,k}\phi_{t-1}))}
{   \sum_{k=1}^K(\psi(\sum_{v=1}^V(\beta_{t,k,v}\phi_{t-1}+n_{t,v,k})) -\psi(\sum_{v=1}^V\beta_{t,k,v}\phi_{t-1}))\phi_{t-1}}.     
\end{equation}

By iterating Gibbs sampling with~\eqref{topic_update_rule} and~\eqref{author_update_rule} and maximum likelihood estimation with~\eqref{dat_alpha1} and~\eqref{eq:beta1}, we can estimate the latent topics $\vec{z}$, authors $\vec{a}$, and parameters $\alpha$ and $\beta$ for each time point.

After the iteration the means of $\phi_{k,v,t}$ and $\theta_{a,k,t}$ can be obtained based on \eqref{dat_phi1} and \eqref{dat_theta1}, these estimation are used to estimate the news trend for the next time period $t+1$.

\begin{equation}\label{dat_phi1}
\phi_{k,v,t} = \left(\frac{n_{t,v,k}+\beta_{t,k,v}\phi_{t-1}}{\sum_{v=1}^V \left(n_{t,v,k}+\beta_{t,v,k}\phi_{t-1} \right) } \right).
\end{equation}
\begin{equation}\label{dat_theta1}
\theta_{a,k,t} = \left(\frac{n_{t,a,k}+\alpha_{t,k} \Theta_{t-1}}{\sum_{k=1}^K \left(n_{t,a,k}+\alpha_{t,k}\theta_{t-1} \right) } \right).
\end{equation}


From the perspective of engineering system, we only use the current data when inferring the current author-topic and topic-word distribution $\boldsymbol{\Theta_{a,k}}$ and $\boldsymbol{\Phi_{w,k}}$, the computational and storage complexity can be largely reduced compared to traditional model which are relied on past data. The processing flow of our DAT model is shown in Figure~\ref{fig:flow}. 


\begin{figure}[h]
\centering
\includegraphics[width=1.1\textwidth]{figures/model_description.png}
\caption{Flow diagram of Dynamic Author-Topic model: an example between time $t-1$ and $t$}
\label{fig:flow}
\end{figure}

So for each time period only the parameters that are needed by the inference for the next period will be remaining with unnecessary data removed. Therefore our model outperforms conventional LDA and Author-Topic model in terms of time and memory efficiency.

The Gibbs Sampler can be realized with the procedure in Algorithm 1, the Gibbs sampling algorithm runs over the three periods of initialization, burn-in and sampling. We assume that our document corpus is a stream of news spanning time period $T$, our Gibbs sampler is ran for each time point $t$ where $t \in [1,T]$. Before the first set of news come into our processing module, namely at time $t = 0$, no previous information is available so all parameters are initialized at this time. After that, our author-topic and topic-word distribution as well as the precision values will be updated based on those of last time. Here $M_t$ represents the total number of news at time period $t$, while $N_{m,t}$ is the total number of unique words from $M_t$ number of news.

\begin{algorithm}\label{algo:ATOT}
\DontPrintSemicolon
\LinesNumbered
 
    \SetKwInOut{Input}{Input}
    \SetKwInOut{Output}{Output}
    \SetKwInOut{Parameter}{Global Data}

    \Input{word vector $\boldsymbol{w}$, author vector $\boldsymbol{a}$, $\alpha$,$\beta$, topic number $K$}
    \Parameter{count statistics $\{n_{a,k}\}$, $\{n_{k,v}\}$, and their sums $\{n_{a}\}$, $\{n_{k}\}$}
    \Output{topic associations $\boldsymbol{z}$, author associations $\boldsymbol{a}$, multinomila parameter $\boldsymbol{\Phi}$, and $\boldsymbol{\Theta}$}
    \If{$t=0$}{
    Set initial values for $\boldsymbol{\Theta}$ as $1/K$ and $\boldsymbol{\Phi}$ as $1/V$, also for $\boldsymbol{\alpha}$ and $\boldsymbol{\beta}$  
    }
    \For{time $t \in [1,T]$}{
    {//  
    \textbf{Initialization}:}\;
    zero all count variables: $\{n_{a,k}\}$, $\{n_{k,v}\}$, $\{n_{a}\}$, $\{n_{k}\}$\, 
    \For{all news $m \in [1,M_t]$ }{
      \For{all words $n \in [1,N_{m,t}]$ in news m }{
      sample topic index $z_{m,n} = k \sim \text{Multinomial}(1/K)$ \par
      sample author index $x_{m,n} = a \sim \text{Multinomial}(1/A_m)$ \par
      increment author-topic count: $n_{a,k} += 1$\par
      increment author-topic sum: $n_{a} += 1$\par
      increment topic-word count: $n_{k,w_{m,n}} += 1$\par
      increment topic-word count: $n_{k} += 1$\par
      }
   }
   
  
   {// \textbf{Gibbs Sampling}:}\;
    sample over burn-in period and sampling period:\; 
    \caption{Inference for the Dynamic Author-Topic model using Gibbs sampling }
    \While{not finished}{
    \For{all news $m \in [1,M_t]$ }{
      \For{all words $n \in [1,N_{m,t}]$ in news m }{
      // For the current assignment of $k$ and $a$ to the word token $w_{m,n}$ \par
      decrement author-topic count: $n_{a,k} -= 1$\par
      decrement author-topic sum: $n_{a} -= 1$\par
      decrement topic-word count: $n_{k,w_{m,n}} -= 1$\par
      decrement topic-word count: $n_{k} -= 1$\par
      sample author index $\boldsymbol{a} \sim \math{P}({x}_{m, n, t}| \mathbf{z}_t,\mathbf{x}_{\neg(m, n, t)}, \mathbf{a_t},\alpha_t,  \boldsymbol{\Theta}_{t-1}) $ according to~\eqref{author_update_rule} \par
      sample topic index $\boldsymbol{z} \sim \math{P}({z}_{m, n, t}| \mathbf{w}_t,\mathbf{z}_{\neg(m, n, t)},\mathbf{x}_t, \mathbf{a_t},\alpha_t, \beta_t,\boldsymbol{\Phi}_{t-1}, \boldsymbol{\Theta}_{t-1})$ according to~\eqref{topic_update_rule}\par
      // For the new assignment of $k$ and $a$ to the word token $w_{m,n}$ \par
      increment author-topic count: $n_{a,k} += 1$\par
      increment author-topic sum: $n_{a} += 1$\par
      increment topic-word count: $n_{k,w_{m,n}} += 1$\par
      increment topic-word count: $n_{k} += 1$\par
      
      }
   }
   Update prevision value $\boldsymbol{\alpha_t}$ according to~\eqref{dat_alpha1}\par
    Update prevision value $\boldsymbol{\beta_t}$ according to~\eqref{eq:beta1}\par
    }
    \If{converged}{
    Update parameter set $\boldsymbol{\Phi_t}$ according to~\eqref{dat_phi1}\par
    Update parameter set $\boldsymbol{\Theta_t}$ according to~\eqref{dat_theta1}\par
    
    $t= t + 1$
    }
    }
\end{algorithm}







\section{Inference}\label{inferenceofthemodel}

Similar to LDA model, based on the algorithm proposed by \cite{blei2003latent} to calculate the approximate maximum likelihood estimates for $\boldsymbol{\phi}$ as well as the hyperparameter of the prior $\boldsymbol{\theta}$, the Markov Chain Monte Carlo (MCMC) is used for inference, MCMC is an approach to obtain the sample from the complicated probability distributions, to allow a Markov chain to converge to a targeted distribution and the drawing the samples from the Markov chain \cite{gilks1996introducing}. Since the inference of the parameters of the distribution of the model is intractable, so collapsed Gibbs Sampling is used here for the inference in which the next state is arrived by sampling all variables from their distribution sequentially when conditioned on the current values of all other variables and the data.
in the above Gibbs sampling procedure, we need to calculate the conditional distribution first, which is
\begin{equation}
P({z}_{m, n, t},{x}_{m, n, t}| \mathbf{w}_t,\mathbf{z}_{\neg(m, n, t)}, \mathbf{x}_{\neg(m, n, t)},\mathbf{a_t},\alpha_t, \beta_t,\boldsymbol{\Phi}_{t-1}, \boldsymbol{\Theta}_{t-1}),
\label{eq:conditional}
\end{equation}
where $\mathbf{z}_{\neg(m, n, t)}, \mathbf{x}_{\neg(m, n, t)}$ represents the topic, author assignments for all the word tokens in the news corpus except for $w_{m,n}$. We can begin with the joint distribution,
\begin{equation}
P(\mathbf{w}_t, \mathbf{z}_t ,\mathbf{x}_t| \alpha_t, \beta_t,\mathbf{a}, \boldsymbol{\Phi}_{t-1}, \boldsymbol{\Theta}_{t-1}).
\end{equation}
so that we can take advantage of the conjugate priors to simplify the integrals. The symbols used here are all defined in Table~\ref{tab:notation-des}. The whole process of Gibbs sampling derivation for our Dynamic Author-Topic model is listed as follows,
\begin{align*}
\multicolumn{2} =   &  \math{P}({z}_{m, n, t},{x}_{m, n, t}| \mathbf{w}_t,\mathbf{z}_{\neg(m, n, t)}, \mathbf{x}_{\neg(m, n, t)},\mathbf{a_t},\alpha_t, \beta_t,\boldsymbol{\Phi}_{t-1}, \boldsymbol{\Theta}_{t-1})
\displaybreak[3]\\
\propto & \quad P(\mathbf{w}_t, \mathbf{z}_t ,\mathbf{x}_t| \alpha_t, \beta_t,\mathbf{a_t}, \boldsymbol{\Phi}_{t-1}, \boldsymbol{\Theta}_{t-1}), \displaybreak[3]\\
& \hspace{-0.1in}\text{based on the graphical model in Figure~\ref{fig:atot}, and integrate on $\boldsymbol{\Phi}$ and $\boldsymbol{\Theta}$ it becomes}\displaybreak[3]\\
= & \quad  P(\mathbf{w}_t | \mathbf{z}_t, \mathbf{\beta}_t,\boldsymbol{\Phi}_{t-1})\times P(\mathbf{z}_t | \boldsymbol{\Theta}_{t-1}, \alpha_t, \mathbf{x}_t) \times P(\mathbf{x}_t | \mathbf{a}_t) \displaybreak[3]\\
= & \quad \int P(\mathbf{w}_t | \mathbf{z}_t, \boldsymbol{\Phi}_t) P(\boldsymbol{\Phi}_t | \boldsymbol{\Phi}_{t-1}, \beta_t) d\boldsymbol{\Phi}_t \int P(\mathbf{z}_t | \mathbf{x}_t, \boldsymbol{\Theta}_t) P(\boldsymbol{\Theta}_t | \boldsymbol{\Theta}_{t-1}, \alpha_t) d\boldsymbol{\Theta}_t \displaybreak[3]\\
&  \times P(\mathbf{x}_t | \mathbf{a}_t),\displaybreak[3]\\
& \hspace{-0.1in}\text{we then rewrite the probabilistic distribution from the way of vector to that of scalar, }\\
& \hspace{-0.1in}\text{for example, $P(\mathbf{w}_t | \mathbf{z}_t, \boldsymbol{\Phi}_t)$ can become the the product of $P({w}_{m, n, t} | \phi_{t, z_{m,n}})$      across the}\\
& \hspace{-0.1in}\text{whole corpus, it becomes,}\displaybreak[3]\\
= & \quad \int \prod_{m=1}^{M} \prod_{n=1}^{N_m} P({w}_{m, n, t} | \phi_{t, z_{m,n}}) \prod_{k=1}^K P(\phi_{t, k} | \phi_{t-1,k}, \beta_t) d\Phi_t \displaybreak[3]\\
&  \times \int \prod_{m=1}^{M} \prod_{n=1}^{N_m} P({z}_{m, n, t} | {\theta}_{x_{m,n}},t) \prod_{a=1}^{A}P(\theta_{a,t}|\theta_{a,t-1},\alpha_t) d\Theta_t \displaybreak[3]\\
&  \times \prod_{m=1}^{M} \prod_{n=1}^{N} P({x}_{m, n, t} | {a}_{m,t}), \displaybreak[3]\\
& \hspace{-0.1in}\text{since extract the word $w_{m,n}$ from topic $z_{n,n}$ exactly follow the distribution of $\phi_{z_{m,n}}$, and }\displaybreak[3]\\
& \hspace{-0.1in}\text{also extract topic $z_{m,n}$ from the author $x_{m,n}$ similarly follow the distribution of $\theta_{x_{m,n}}$, so, }\displaybreak[3]\\
= & \quad \int \prod_{k=1}^K \prod_{v=1}^V \phi_{t, v, k}^{n_{(t, v,k)}} \prod_{k=1}^K P(\phi_{t, k} | \phi_{t-1, k}, \beta_t) d\Phi_t \displaybreak[3]\\
&  \times \int \prod_{a=1}^A \prod_{k=1}^K \theta_{t, a, k}^{n_{(t, a,k)}} \prod_{a=1}^A P(\theta_{a,t}|\theta_{a,t-1},\alpha_t) d\Theta_t \times \left(\frac{1}{\prod_{m=1}^M A_m^{N_m}} \right), \displaybreak[3]\\
%
& \hspace{-0.1in}\text{based on the posterior of  Dirichlet distribution \cite{ferguson1973bayesian}, }\displaybreak[3]\\
= & \quad \int \prod_{k=1}^K \prod_{v=1}^V \phi_{t,v,k}^{n_{(t,v,k)}} \prod_{k=1}^K \left( \frac{\mathrm{\Gamma} (\sum_{v=1}^V \beta_{(t, k, v)} {\phi_{t-1}})}{\prod_{v=1}^V \mathrm{\Gamma}(\beta_{(t, k, v)} {\phi_{t-1}})} \prod_{v=1}^V \phi_{t, k, v}^{\left(\beta_{(t, k, v)} {\phi_{t-1}} \right) -1} \right) d\Phi_t \displaybreak[3]\\
%
&  \times \int \prod_{a=1}^A \prod_{k=1}^K \theta_{t, a, k}^{n_{(t, a,k)}} \prod_{a=1}^A \left( \frac{\mathrm{\Gamma}(\sum_{k=1}^k \alpha_{(t, k)} \theta_{t-1, k})}{\prod_{k=1}^K \mathrm{\Gamma} (\alpha_{(t, k)} \theta_{t-1, k})} \prod_{k=1}^K \Theta_{t, k}^{\left(\alpha_{(t, k)} {\Theta_{t-1,k}} \right) -1}  \right)  d\Theta_t \displaybreak[3]\\
%
&  \times \left(\frac{1}{\prod_{m=1}^M A_m^{N_m}} \right), \displaybreak[3]\\
%
%
%= & \prod_{z=1}^Z \frac{\mathrm{\Gamma} (\sum_{v=1}^V \beta_{t, z, v} \abbrev{\phi})}{\prod_{v=1}^V \mathrm{\Gamma}(\beta_{t, z, v} \abbrev{\phi})} \int \prod_{z=1}^Z \prod_{v=1}^V \phi_{t, z, v}^{n_{t, z, v}} \prod_{z=1}^Z \prod_{v=1}^V \phi_{t, z, v}^{\beta_{t, z, v} \abbrev{\phi} -1} d\Phi_t \displaybreak[3]\\
%& \times \frac{\mathrm{\Gamma}(\sum_{z=1}^Z \alpha_{t, z} \theta_{t-1, z})}{\prod_{z=1}^Z \mathrm{\Gamma} (\alpha_{t, z} \theta_{t-1, z})} \int \prod_{z=1}^Z \theta_{t, z}^{m_{t, z}}  \prod_{z=1}^Z \theta_{t, z}^{\alpha_{t, z} \theta_{t-1, z} -1} d\Theta_t \displaybreak[3]\\
%
& \hspace{-0.1in}\text{since for different topics, based on d-separation \cite{geiger2013d} their topic-word distribution can be  }\displaybreak[3]\\
& \hspace{-0.1in}\text{regarded as independent from each other, which is the same for author-topic distributions,  }\displaybreak[3]\\
& \hspace{-0.1in}\text{so that the integration of products can be rewritten as product of integrations,}\displaybreak[3]\\
= & \quad \prod_{k=1}^K \frac{\mathrm{\Gamma} (\sum_{v=1}^V \beta_{(t, k, v)} {\phi_{t-1}})}{\prod_{v=1}^V \mathrm{\Gamma}(\beta_{(t, k, v)} {\phi_{t-1}})} \prod_{k=1}^K \int \prod_{v=1}^V \phi_{t, v,k}^{n_{t, v,k}+\beta_{t, k,v} {\phi_{t-1}} -1} d\Phi_t \displaybreak[3]\\
& \times \prod_{a=1}^A \frac{\mathrm{\Gamma}(\sum_{k=1}^K \alpha_{(t, k)} \theta_{t-1, k})}{\prod_{k=1}^K \mathrm{\Gamma} (\alpha_{(t, k)} \theta_{(t-1, k)})} \prod_{a=1}^A \int \prod_{k=1}^K \theta_{t, k}^{n_{t,a,k}+\alpha_{t, k} \theta_{t-1, k} -1} d\Theta_t \displaybreak[3]\\
& \times \left(\frac{1}{\prod_{m=1}^M A_m^{N_m}} \right), \displaybreak[3]\\
& \hspace{-0.1in}\text{according to Euler integral \cite{jeffrey2008handbook},}\displaybreak[3]\\
= & \quad \prod_{k=1}^K \frac{\mathrm{\Gamma} (\sum_{v=1}^V \beta_{t, k, v} {\phi_{t-1}})}{\prod_{v=1}^V \mathrm{\Gamma}(\beta_{t, k, v} {\phi_{t-1}})} \prod_{k=1}^K \frac{\prod_{v=1}^V \mathrm{\Gamma} (n_{t,v,k} + \beta_{t, k, v} {\phi_{t-1}})}{\mathrm{\Gamma} (\sum_{v=1}^V n_{t,v,k} + \beta_{t, k, v} {\phi_{t-1}})} \displaybreak[3]\\
&  \times \prod_{a=1}^A \frac{\mathrm{\Gamma}(\sum_{k=1}^K \alpha_{t, k} \theta_{t-1, k})}{\prod_{k=1}^K \mathrm{\Gamma} (\alpha_{t, k} \theta_{t-1, k})}  \prod_{a=1}^A \frac{\prod_{k=1}^K \mathrm{\Gamma} (n_{t,a,k} + \alpha_{t, k} \theta_{t-1, k})}{\mathrm{\Gamma} (\sum_{k=1}^K n_{t,a, k} + \alpha_{t, k} \theta_{t-1, k})} \displaybreak[3]\\
& \times   \left(\frac{1}{\prod_{m=1}^M A_m^{N_m}} \right), \displaybreak[3]\\
%
\end{align*}
where $n_{t,a,k} = {n_{a,t}^{(k)}} $ representing the number of times the author $a$ is assigned to topic $k$, at time $t$, and $n_{t,v,k}$ representing the number of times word token $v$ is assigned to topic $k$, at time $t$.
Applying the chain rule, we can obtain the conditional probability,

\begin{align*}
\multicolumn{2} =   &  \math{P}({z}_{m, n, t},{x}_{m, n, t}| \mathbf{w}_t,\mathbf{z}_{\neg(m, n, t)}, \mathbf{x}_{\neg(m, n, t)},\mathbf{a_t},\alpha_t, \beta_t,\boldsymbol{\Phi}_{t-1}, \boldsymbol{\Theta}_{t-1})
\displaybreak[3]\\
= & \quad \frac{\math{P}({w}_{m, n,t},{z}_{m, n, t},{x}_{m, n, t}| \mathbf{w}_{\neg(m, n, t)},\mathbf{z}_{\neg(m, n, t)}, \mathbf{x}_{\neg(m, n, t)},\mathbf{a_t},\alpha_t, \beta_t,\boldsymbol{\Phi}_{t-1}, \boldsymbol{\Theta}_{t-1})}{\math{P}({w}_{m, n, t}| \mathbf{w}_{\neg(m, n, t)},\mathbf{z}_{\neg(m, n, t)}, \mathbf{x}_{\neg(m, n, t)},\mathbf{a_t},\alpha_t, \beta_t,\boldsymbol{\Phi}_{t-1}, \boldsymbol{\Theta}_{t-1})}
\displaybreak[3]\\
= & \quad \frac{P(\mathbf{w}_t, \mathbf{z}_t,\mathbf{x}_t |  \alpha_t, \beta_t, \boldsymbol{\Phi}_{t-1}, \boldsymbol{\Theta}_{t-1})}{P(\mathbf{w}_{t}, \mathbf{z}_{t,\neg(m,n)},\mathbf{x}_{t,\neg(m,n)} | \alpha_t, \beta_t,a, \boldsymbol{\Phi}_{t-1}, \boldsymbol{\Theta}_{t})} \displaybreak[3]\\
%
= & \quad \frac{P(\mathbf{w}_t, \mathbf{z}_t,\mathbf{x}_t |  \alpha_t, \beta_t, \boldsymbol{\Phi}_{t-1}, \boldsymbol{\Theta}_{t-1})}{P(\mathbf{w}_{t,\neg(m,n)}, \mathbf{z}_{t,\neg(m,n)},\mathbf{x}_{t,\neg(m,n)} | \alpha_t, \beta_t,a, \boldsymbol{\Phi}_{t-1}, \boldsymbol{\Theta}_{t})P(w_{m,n,t} |  \alpha_t, \beta_t, \boldsymbol{\Phi}_{t-1}, \boldsymbol{\Theta}_{t-1})} \displaybreak[3]\\
\propto & \quad \frac{P(\mathbf{w}_t, \mathbf{z}_t,\mathbf{x}_t |  \alpha_t, \beta_t, \boldsymbol{\Phi}_{t-1}, \boldsymbol{\Theta}_{t-1})}{P(\mathbf{w}_{t,\neg(m,n)}, \mathbf{z}_{t,\neg(m,n)},\mathbf{x}_{t,\neg(m,n)} | \alpha_t, \beta_t,a, \boldsymbol{\Phi}_{t-1}, \boldsymbol{\Theta}_{t})} \displaybreak[3]\\
\propto & \quad \prod_{k=1}^K \frac{\mathrm{\Gamma} (\sum_{v=1}^V \beta_{t, k, v} {\phi_{t-1}})}{\prod_{v=1}^V \mathrm{\Gamma}(\beta_{t, k, v} {\phi_{t-1}})} \prod_{k=1}^K \frac{\prod_{v=1}^V \mathrm{\Gamma} (n_{t,v,k} + \beta_{t, k, v} {\phi_{t-1}})}{\mathrm{\Gamma} (\sum_{v=1}^V n_{t,v,k} + \beta_{t, k, v} {\phi_{t-1}})} \displaybreak[3]\\
&  \times \prod_{a=1}^A \frac{\mathrm{\Gamma}(\sum_{k=1}^K \alpha_{t, k} \theta_{t-1, k})}{\prod_{k=1}^K \mathrm{\Gamma} (\alpha_{t, k} \theta_{t-1, k})}  \prod_{a=1}^A \frac{\prod_{k=1}^K \mathrm{\Gamma} (n_{t,a,k} + \alpha_{t, k} \theta_{t-1, k})}{\mathrm{\Gamma} (\sum_{k=1}^K n_{t,a, k} + \alpha_{t, k} \theta_{t-1, k})}, \displaybreak[3]\\
& \hspace{-0.1in}\text{applying $\mathrm{\Gamma}(x)=(x-1)\mathrm{\Gamma}(x-1)$ and $\mathrm{\Gamma}(x+m)=\prod_{i=1}^m (x+i-1)\mathrm{\Gamma}(x)$, and considering } \displaybreak[3]\\
& \hspace{-0.1in}\text{that author $a$ is associated with its own topic $z$, it becomes} \displaybreak[3]\\
\propto & \quad  \left(\frac{n_{t,v,k}+\beta_{t,k,v}\phi_{t-1}-1}{\sum_{v=1}^V \left(n_{t,v,k}+\beta_{t,v,k}\phi_{t-1} \right)-1 } \right) \times   \left(\frac{n_{t,a,k}+\alpha_{t,k} \Theta_{t-1}-1}{\sum_{k=1}^K \left(n_{t,a,k}+\alpha_{t,k}\theta_{t-1} \right)-1 } \right).
\displaybreak[3]\\
\end{align*}

If we manipulate the above formula to turn the above update equations for the topic and author of each token into separated updated ones, which can obtain the following update rules which are suitable for random or systematic scan updates, 
\begin{itemize}
  \item For the topic:
  \begin{align*}
\multicolumn{2} =   &  \math{P}({z}_{m, n, t}| \mathbf{w}_t,\mathbf{z}_{\neg(m, n, t)},\mathbf{x}_t, \mathbf{a_t},\alpha_t, \beta_t,\boldsymbol{\Phi}_{t-1}, \boldsymbol{\Theta}_{t-1})
\displaybreak[3]\\
= & \quad \frac{\math{P}({w}_{m, n, t},{z}_{m, n, t}| \mathbf{w}_{\neg(m, n, t)},\mathbf{z}_{\neg(m, n, t)},\mathbf{x}_t, \mathbf{a_t},\alpha_t, \beta_t,\boldsymbol{\Phi}_{t-1}, \boldsymbol{\Theta}_{t-1})}{\math{P}({w}_{m, n, t}| \mathbf{w}_{\neg(m, n, t)},\mathbf{z}_{\neg(m, n, t)},\mathbf{x}_t, \mathbf{a_t},\alpha_t, \beta_t,\boldsymbol{\Phi}_{t-1}, \boldsymbol{\Theta}_{t-1})}
\displaybreak[3]\\
= & \quad \frac{P(\mathbf{w}_t, \mathbf{z}_t | \mathbf{x}_t, \alpha_t, \beta_t, \boldsymbol{\Phi}_{t-1}, \boldsymbol{\Theta}_{t-1})}{P(\mathbf{w}_{t}, \mathbf{z}_{t,\neg(m,n)} |\mathbf{x}_t, \alpha_t, \beta_t,\mathbf{a_t}, \boldsymbol{\Phi}_{t-1}, \boldsymbol{\Theta}_{t})} \displaybreak[3]\\
= & \quad \frac{P(\mathbf{w}_t, \mathbf{z}_t | \mathbf{x}_t, \alpha_t, \beta_t, \boldsymbol{\Phi}_{t-1}, \boldsymbol{\Theta}_{t-1})}{P(\mathbf{w}_{t,\neg(m,n)}, \mathbf{z}_{t,\neg(m,n)} | \mathbf{x}_{t},\alpha_t, \beta_t,\mathbf{a_t}, \boldsymbol{\Phi}_{t-1}, \boldsymbol{\Theta}_{t})P(w_{m,n} | \mathbf{x}_{t}, \alpha_t, \beta_t, \boldsymbol{\Phi}_{t-1}, \boldsymbol{\Theta}_{t-1})} \displaybreak[3]\\
\propto & \quad \frac{P(\mathbf{w}_t, \mathbf{z}_t | \mathbf{x}_t, \alpha_t, \beta_t, \boldsymbol{\Phi}_{t-1}, \boldsymbol{\Theta}_{t-1})}{P(\mathbf{w}_{t,\neg(m,n)}, \mathbf{z}_{t,\neg(m,n)} | \mathbf{x}_{t},\alpha_t, \beta_t,\mathbf{a_t}, \boldsymbol{\Phi}_{t-1}, \boldsymbol{\Theta}_{t})} \displaybreak[3]\\
%
\propto & \quad \prod_{k=1}^K \frac{\mathrm{\Gamma} (\sum_{v=1}^V \beta_{t, k, v} {\phi_{t-1}})}{\prod_{v=1}^V \mathrm{\Gamma}(\beta_{t, k, v} {\phi_{t-1}})} \prod_{k=1}^K \frac{\prod_{v=1}^V \mathrm{\Gamma} (n_{t,v,k} + \beta_{t, k, v} {\phi_{t-1}})}{\mathrm{\Gamma} (\sum_{v=1}^V n_{t,v,k} + \beta_{t, k, v} {\phi_{t-1}})} \displaybreak[3]\\
&  \times \prod_{a=1}^A \frac{\mathrm{\Gamma}(\sum_{k=1}^K \alpha_{t, k} \theta_{t-1, k})}{\prod_{k=1}^K \mathrm{\Gamma} (\alpha_{t, k} \theta_{t-1, k})}  \prod_{a=1}^A \frac{\prod_{k=1}^K \mathrm{\Gamma} (n_{t,a,k} + \alpha_{t, k} \theta_{t-1, k})}{\mathrm{\Gamma} (\sum_{k=1}^K n_{t,a, k} + \alpha_{t, k} \theta_{t-1, k})} \displaybreak[3]\\
\propto & \quad  \left(\frac{n_{t,v,k}+\beta_{t,k,v}\phi_{t-1}-1}{\sum_{v=1}^V \left(n_{t,v,k}+\beta_{t,v,k}\phi_{t-1} \right)-1 } \right) \times   \left(\frac{n_{t,a,k}+\alpha_{t,k} \Theta_{t-1}-1}{\sum_{k=1}^K \left(n_{t,a,k}+\alpha_{t,k}\theta_{t-1} \right)-1 } \right).
\displaybreak[3]\\
\end{align*}
  \item For the author:
\begin{align*}
\multicolumn{2} =   &  \math{P}({x}_{m, n, t}| \mathbf{z}_t,\mathbf{x}_{\neg(m, n, t)}, \mathbf{a_t},\alpha_t,  \boldsymbol{\Theta}_{t-1})
\displaybreak[3]\\
= & \quad \frac{\math{P}({x}_{m, n, t},{z}_{m, n, t}| \mathbf{x}_{\neg(m, n, t)},\mathbf{z}_{\neg(m, n, t)}, \mathbf{a_t},\alpha_t, , \boldsymbol{\Theta}_{t-1})}{\math{P}({z}_{m, n, t}| \mathbf{x}_{\neg(m, n, t)},\mathbf{z}_{\neg(m, n, t)}, \mathbf{a_t},\alpha_t,  \boldsymbol{\Theta}_{t-1})}
\displaybreak[3]\\
= & \quad \frac{P(\mathbf{x}_t, \mathbf{z}_t | \mathbf{a_t}, \alpha_t, \boldsymbol{\Theta}_{t-1})}{P(\mathbf{z}_{t}, \mathbf{x}_{t,\neg(m,n)} | \mathbf{a_t},\alpha_t,  \boldsymbol{\Theta}_{t})} \displaybreak[3]\\
= & \quad \frac{P(\mathbf{x}_t, \mathbf{z}_t | \mathbf{a_t}, \alpha_t, \boldsymbol{\Theta}_{t-1})}{P(\mathbf{x}_{t,\neg(m,n)}, \mathbf{z}_{t,\neg(m,n)} | \alpha_t, \mathbf{a_t},  \boldsymbol{\Theta}_{t})P(z_{m,n} |  \alpha_t, \boldsymbol{\Theta}_{t-1})} \displaybreak[3]\\
\propto & \quad \frac{P(\mathbf{x}_t, \mathbf{z}_t |  \alpha_t, \mathbf{a_t}, \boldsymbol{\Theta}_{t-1})}{P(\mathbf{x}_{t,\neg(m,n)}, \mathbf{z}_{t,\neg(m,n)} | \alpha_t, \mathbf{a_t},  \boldsymbol{\Theta}_{t})} \displaybreak[3]\\
%
\propto & \quad  \prod_{a=1}^A \frac{\mathrm{\Gamma}(\sum_{k=1}^K \alpha_{t, k} \theta_{t-1, k})}{\prod_{k=1}^K \mathrm{\Gamma} (\alpha_{t, k} \theta_{t-1, k})}  \prod_{a=1}^A \frac{\prod_{k=1}^K \mathrm{\Gamma} (n_{t,a,k} + \alpha_{t, k} \theta_{t-1, k})}{\mathrm{\Gamma} (\sum_{k=1}^K n_{t,a, k} + \alpha_{t, k} \theta_{t-1, k})} \displaybreak[3]\\
\propto & \quad   \left(\frac{n_{t,a,k}+\alpha_{t,k} \Theta_{t-1}-1}{\sum_{k=1}^K \left(n_{t,a,k}+\alpha_{t,k}\theta_{t-1} \right)-1 } \right).
\displaybreak[3]\\
\end{align*}
  
\end{itemize}

Based on the above inference, the multinomila parameter sets $\Theta$ and $\Phi$ can be obtained, according to the definitions as multinomial distributions with Dirichlet prior, to apply the Bayes' rule we will get,
\begin{align*} \label{theta}
P(\boldsymbol{\Theta_{a,t}}|\mathbf{z_t},\mathbf{x_t},\mathbf{\alpha_t},\boldsymbol{\Theta_{a,t-1}}) & = \frac{P(\boldsymbol{\Theta_{a,t}},\mathbf{z_t}|\mathbf{x_t},\mathbf{\alpha_t},\boldsymbol{\Theta_{a,t-1}})}{P(\mathbf{z_t}|\mathbf{x_t},\mathbf{\alpha_t},\boldsymbol{\Theta_{a,t-1}})} \\
 & = {\frac{1}{Z_\boldsymbol{\Theta_{a,t}}}}\prod_{(m,n):x_{m,n}=a}P(z_{m,n}|\boldsymbol{\Theta}_{a,t})P(\boldsymbol{\Theta}_{a,t}|\alpha_t,\boldsymbol{\Theta}_{a,t-1})\displaybreak[3]\\
 & = {\frac{1}{Z_\boldsymbol{\Theta_{a,t}}}}\prod_{k=1}^{K}\theta_{a,k}^{n_{a,k}} \times \frac{\mathrm{\Gamma}(\sum_{k=1}^K \alpha_{t, k} \theta_{t-1, k})}{\prod_{k=1}^K \mathrm{\Gamma} (\alpha_{t, k} \theta_{t-1, k})} \prod_{k=1}^{K}\theta_{a,k}^{\alpha_{t,k}\theta{(t-1),k,a}-1}\\
 & = {\frac{1}{Z_\boldsymbol{\Theta_{a,t}}}}\prod_{k=1}^{K}\theta_{a,k}^{n_{a,k}+\alpha_{t,k}\theta{(t-1),k,a}-1}\\
& = \text{Dirichelet}{(\boldsymbol{\Theta}_{a,t}|\boldsymbol{n}_{a,t} + \boldsymbol{\alpha_t}\boldsymbol{\Theta}_{a,t-1})},
\end{align*}
where $\boldsymbol{n}_a = \{n_a^{(k)}\}_{k=1}^K$ is the vector of all the topic observations counts for a specific author $a$, while for $\boldsymbol{\Phi}$, it is,

\begin{align*} \label{phi}
P(\boldsymbol{\Phi_{k,t}}|\mathbf{z_t},\mathbf{w_t},\mathbf{\beta_t},\boldsymbol{\Phi_{k,t-1}}) & = \frac{P(\boldsymbol{\Phi_{k,t}},\mathbf{w_t}|\mathbf{z_t},\mathbf{\beta_t},\boldsymbol{\Phi_{k,t-1}})}{P(\mathbf{w_t}|\mathbf{z_t},\mathbf{\beta_t},\boldsymbol{\Phi_{k,t-1}})} \\
 & = {\frac{1}{Z_\boldsymbol{\Phi_{k,t}}}}\prod_{(m,n):z_{m,n}=k}P(w_{m,n}|\boldsymbol{\Phi}_{k,t})P(\boldsymbol{\Phi}_{k,t}|\beta_t,\boldsymbol{\Phi}_{k,t-1})\displaybreak[3]\\
 & = {\frac{1}{Z_\boldsymbol{\Phi_{k,t}}}}\prod_{v=1}^{V}\phi_{t,v}^{n_{k,v}} \times \frac{\mathrm{\Gamma}(\sum_{v=1}^V \beta_{v,t} \phi_{t-1,k})}{\prod_{v=1}^V \mathrm{\Gamma} (\beta_{t, v} \phi_{t-1,k})} \prod_{v=1}^{V}\phi_{k,v}^{\beta_{v,t}\phi{(t-1),k,v}-1}\\
 & = {\frac{1}{Z_\boldsymbol{\Phi_{k,t}}}}\prod_{v=1}^{V}\phi_{k,v}^{n_{k,v}+\beta_{t,v}\phi{(t-1),k,v}-1}\\
& = \text{Dirichelet}{(\boldsymbol{\Phi}_{k,t}|\boldsymbol{n}_{k,t} + \boldsymbol{\beta}_t{\boldsymbol{\Phi}_{k,t-1}})},
\end{align*}
where $\boldsymbol{n}_k = \{n_k^{(v)}\}_{v=1}^V$ is the vector of all the word observations counts for a specific topic $k$. By calculating the expectations of the Dirichelet distribution on the above two equations it will yield:
\begin{equation}\label{dat_phi}
\phi_{k,v,t} = \left(\frac{n_{t,v,k}+\beta_{t,k,v}\phi_{t-1}}{\sum_{v=1}^V \left(n_{t,v,k}+\beta_{t,v,k}\phi_{t-1} \right) } \right).
\end{equation}
\begin{equation}\label{dat_theta}
\theta_{a,k,t} = \left(\frac{n_{t,a,k}+\alpha_{t,k} \Theta_{t-1}}{\sum_{k=1}^K \left(n_{t,a,k}+\alpha_{t,k}\theta_{t-1} \right) } \right).
\end{equation}
\eqref{dat_phi} and~\eqref{dat_theta} are the update rules for $\boldsymbol{\Phi}$ and $\boldsymbol{\Theta}$.

Then we will show how to update $\alpha$ and $\beta$ for each time frame, by maximizing the joint distribution $ P(\mathbf{w}_t, \mathbf{z}_t ,\mathbf{x}_t| \alpha_t, \beta_t,a, \boldsymbol{\Phi}_{t-1}, \boldsymbol{\Theta}_{t-1})$ and applying fixed-point iteration for estimation. The steps are as follows,
\begin{align*}
\multicolumn{2}=   &   P(\mathbf{w}_t, \mathbf{z}_t ,\mathbf{x}_t| \alpha_t, \beta_t,\mathbf{a}, \boldsymbol{\Phi}_{t-1}, \boldsymbol{\Theta}_{t-1}) \displaybreak[3]\\
= & \quad \frac{1}{\prod_{m=1}^M A_{m}^{N_m}} \times  \prod_{k=1}^K  \frac{\mathrm{\Gamma} (\sum_{v=1}^V\beta_{t, k, v} {\phi_{t-1}})}{\prod_{v=1}^V \mathrm{\Gamma}\beta_{t, k, v} {\phi_{t-1}}}  \times \prod_{k=1}^K \frac{\prod_{v=1}^V \mathrm{\Gamma} (n_{t,v,k} + \beta_{t, k, v} {\phi_{t-1}})}{\mathrm{\Gamma} (\sum_{v=1}^V n_{t,v,k} + \beta_{t, k, v} {\phi_{t-1}})} \displaybreak[3]\\
& \times  \quad \prod_{a=1}^A \frac{\mathrm{\Gamma}(\sum_{k=1}^K \alpha_{t, k} \theta_{t-1, k})}{\prod_{k=1}^K \mathrm{\Gamma} (\alpha_{t, k} \theta_{t-1, k})}  \times \prod_{a=1}^A \frac{\prod_{k=1}^K \mathrm{\Gamma} (n_{t,a,k} + \alpha_{t, k} \theta_{t-1, k})}{\mathrm{\Gamma} (\sum_{k=1}^K n_{t,a, k} + \alpha_{t, k} \theta_{t-1, k})}, \displaybreak[3]\\
& \hspace{-0.1in}\text{by applying logarithmic on both sides, it becomes,}\displaybreak[3]\\
\multicolumn{2} = & \log P(\mathbf{w}_t, \mathbf{z}_t ,\mathbf{x}_t| \alpha_t, \beta_t,\mathbf{a}, \boldsymbol{\Phi}_{t-1}, \boldsymbol{\Theta}_{t-1}) \displaybreak[3]\\
%
= & \quad \log \frac{1}{\prod_{m=1}^M A_{m}^{N_m}} + \prod_{k=1}^K \log{ \frac{\mathrm{\Gamma} (\sum_{v=1}^V\beta_{t, k, v} {\phi_{t-1}})}{\prod_{v=1}^V \mathrm{\Gamma}\beta_{t, k, v} {\phi_{t-1}}} } + \prod_{k=1}^K \log{ \frac{\prod_{v=1}^V \mathrm{\Gamma} (n_{t,v,k} + \beta_{t, k, v} {\phi_{t-1}})}{\mathrm{\Gamma} (\sum_{v=1}^V n_{t,v,k} + \beta_{t, k, v} {\phi_{t-1}})} }  \displaybreak[3]\\
& + \quad \prod_{a=1}^A \log{\frac{\mathrm{\Gamma}(\sum_{k=1}^K \alpha_{t, k} \theta_{t-1, k})}{\prod_{k=1}^K \mathrm{\Gamma} (\alpha_{t, k} \theta_{t-1, k})}} + \prod_{a=1}^A \log{\frac{\prod_{k=1}^K \mathrm{\Gamma} (n_{t,a,k} + \alpha_{t, k} \theta_{t-1, k})}{\mathrm{\Gamma} (\sum_{k=1}^K n_{t,a, k} + \alpha_{t, k} \theta_{t-1, k})}} \displaybreak[3]\\
%
=  & \quad C + \sum_{a=1}^A \log{\mathrm{\Gamma}(\sum_{k=1}^K \alpha_{t, k} \theta_{t-1, k})} + \sum_{a=1}^A \sum_{k=1}^K \log{ \mathrm{\Gamma} (n_{t,a,k} + \alpha_{t, k} \theta_{t-1, k})}
\displaybreak[3]\\
&-  \quad \sum_{a=1}^A \sum_{k=1}^K \log{\mathrm{\Gamma} (\alpha_{t, k} \theta_{t-1, k})} - \sum_{a=1}^A \log{\mathrm{\Gamma}(\sum_{k=1}^K (n_{t,a,k} + \alpha_{t, k} \theta_{t-1, k}))}. \displaybreak[3]\\
\end{align*}
Using the bounds~\cite{minka2000estimating}: for any $x^* \in \mathbb{R}^{+}$, $n \in \mathbb{Z}^+$ and $x^*$'s estimation $x$:
%
\begin{align*}
\log \mathrm{\Gamma} (x^*) - \log \mathrm{\Gamma} (x^*+n) \geq & \log \mathrm{\Gamma} (x) - \log \mathrm{\Gamma} (x + n) + \left( \Psi(x + n) - \Psi(x) \right) (x - x^*),
\end{align*}
%
and 
%
\begin{align*}
\log \mathrm{\Gamma} (x^* + n) - \log \mathrm{\Gamma} (x^*) \geq & \log \mathrm{\Gamma} (x + n) - \log \mathrm{\Gamma} (x)  + x \left( \Psi(x + n) - \Psi(x) \right) (\log x^* - \log x),
\end{align*}
we know  $\alpha_{t, k}^*$ should be the optimal parameter in the next fixed-point iteration, notice that $C$ is the function with no relation to $\alpha$ and  $C'$ is the function with no relation to $\alpha^*$, both will be integrated out by taking $\frac{\partial (\cdot)}{\partial \alpha_{t, z}^*}$ to $\alpha_{t, z}^*$. And in the following inference $\Psi(\cdot)$ is the digamma function defined by $\Psi(x)=\frac{\partial \log \mathrm{\Gamma}(x)}{\partial x}$, so it becomes,
\begin{align*}
& \log P(\mathbf{w}_t, \boldsymbol{z}_t ,\boldsymbol{x}_t| \lbrace{\alpha_{t,1,...} \alpha^*_{t,k,...}}\rbrace, \beta_t,a, \boldsymbol{\Phi}_{t-1}, \boldsymbol{\Theta}_{t-1}) \displaybreak[3]\\
%
\geq & \quad Boundary (\alpha^*_{t,k}) \displaybreak[3]\\
%
= & \quad C + C'- \sum_{a=1}^A\alpha^*_{t, k} \theta_{t-1, k}(\psi(\sum_{k=1}^K (\alpha_{t, k} \theta_{t-1, k} + n_{t,a}) - \psi(\sum_{k=1}^K \alpha_{t, k} \theta_{t-1, k})) \displaybreak[3]\\
& + \quad \sum_{a=1}^A\alpha_{t, k} \theta_{t-1, k}(\psi (n_{t,a,k}+\alpha_{t, k} \theta_{t-1, k}  ) - \psi( \alpha_{t, k} \theta_{t-1, k})) \log{\alpha^*_{t, k}\theta_{t-1, k}},\displaybreak[3]\\
\end{align*}
By taking $\frac{\partial (\cdot)}{\partial \alpha_{t, z}^*}$ to $\alpha_{t, z}^*$, we will get,

\begin{align*}
\frac{\partial(B(\alpha^*_{t,k}))}{\partial \alpha^*_{t,k}} = & \frac{\sum_{a=1}^A\alpha_{t, k} \theta_{t-1, k}(\psi(n+\alpha \Theta))
 - \psi(\alpha \Theta)}{\log{\Theta_{t-1,k}\alpha^*_{t,k}}}
\displaybreak[3]\\
& - \sum_{a=1}^A(\psi(\sum_{k=1}^K\alpha \Theta +n)) -\psi(\sum_{k=1}^K\alpha \Theta) \displaybreak[3]\\
= & 0,  \displaybreak[3]\\
\end{align*}
so that we can get,
\begin{equation}\label{dat_alpha}
 \quad \alpha^*_{t,k}= \frac{\sum_{a=1}^A \alpha_{t, k}(\psi(n_{a,t,k}+\alpha_{t, k} \theta_{t-1, k})-\psi(\alpha_{t,k}\theta_{t-1, k}))}{\sum_{a=1}^A (\psi(\sum_{k=1}^K(\alpha_{t, k} \theta_{t-1, k}+n_{t,a,k}))-\psi(\sum_{k=1}^K(\alpha_{t, k} \theta_{t-1, k})))}.  \displaybreak[3]\\
\end{equation}

Similarly, we can deduct $\beta_{t,k,v}^*$ by supposing it is the optimal parameter in the next fixed-point iteration, we will have, 
\begin{align*}
 & P(\mathbf{w}_t, \mathbf{z}_t ,\mathbf{x}_t| \alpha_t, \beta_t,\mathbf{a}, \boldsymbol{\Phi}_{t-1}, \boldsymbol{\Theta}_{t-1}) \displaybreak[3]\\
= & \quad C+ \sum_{k=1}^K \log{\mathrm{\Gamma}(\sum_{v=1}^V\beta_{t, k, v} {\phi_{t-1}})} + \sum_{k=1}^K \sum_{v=1}^V \log{\mathrm{\Gamma}(n_{t,v,k}+\beta_{t,v,k}\phi_{t-1})} \displaybreak[3]\\
& -  \quad \sum_{k=1}^K \sum_{v=1}^V \log{\mathrm{\Gamma}(\beta_{t,v,k}\phi_{t-1})} -\sum_{k=1}^K\log{\mathrm{\Gamma}(\sum_{v=1}^V(\beta_{t, k, v} \phi_{t-1}+n_{t,v,k}))} \displaybreak[3]\\
\multicolumn{3}{l}{$ \log{ P(\mathbf{w}_t, \mathbf{z}_t ,\mathbf{x}_t| \lbrace{\beta_{t,k,v}... \beta^*_{t,k,v}...}\rbrace, \alpha_t,a, \mathbf{\Phi}_{t-1}, \mathbf{\Theta}_{t-1})}  $} \displaybreak[3]\\
\geq & \quad Boundary (\beta^*_{t,k,v}) \displaybreak[3]\\
= & \quad  C + C' - \sum_{k=1}^K(\psi(\sum_{v=1}^V(\beta\phi+n))-\psi(\sum_{v=1}^V\beta\phi))\beta^*\phi \displaybreak[3]\\
& +  \sum_{k=1}^K\sum_{v=1}^V \beta\phi(\psi(\beta\phi+n)-\psi(\beta\phi))\log{\beta^*\phi}, \displaybreak[3]\\
\end{align*}
Now $C$ and $C'$ are the functions with no relations with $\beta$ and $\beta^*$ respectively. By taking $\frac{\partial (\cdot)}{\partial \beta{t, z}^*}$ to $\beta{t, z}^*$, we will get,
\begin{align*}
 \frac{\partial(B(\beta^*_{t,k,v}))}{\partial \beta^*_{t,k,v}} = & \frac{\sum_{k=1}^K\sum_{v=1}^V \beta\phi(\psi(\beta\phi+n)-\psi(\beta\phi))}{\log{\phi\beta^*}}  \displaybreak[3]\\
- & \sum_{k=1}^K((\psi(\sum_{v=1}^V(\beta\phi+n))-\psi(\sum_{v=1}^V\beta\phi))\Phi \displaybreak[3]\\
= & 0,
\displaybreak[3]\\
\end{align*}
so we can get the update rules for $\beta$, as in~\eqref{eq:beta}
\begin{equation}\label{eq:beta}
 \beta^*_{t,k,v}=\frac{\sum_{k=1}^K \beta_{t,v,k}\phi_{t-1}(\psi(\beta_{t,k,v}\phi_{t-1}+n_{t,v,k})-\psi(\beta_{t,v,k}\phi_{t-1}))}
{   \sum_{k=1}^K(\psi(\sum_{v=1}^V(\beta_{t,k,v}\phi_{t-1}+n_{t,v,k})) -\psi(\sum_{v=1}^V\beta_{t,k,v}\phi_{t-1}))\phi_{t-1}}.     
\end{equation}
\eqref{dat_phi},~\eqref{dat_theta},~\eqref{dat_alpha} and~\eqref{eq:beta} consist the update rules which will be applied on the Gibbs sampling algorithm.

In this chapter we introduce the Dynamic Author-Topic model includes motivation, task and inference process. In the next chapter we will give an introduction of the experimental setting.








